%!TEX root = main.tex
\section{Introducción}
La contaminación ambiental es un problema global que afecta la calidad de vida de las personas y el equilibrio ecológico del planeta. La capacidad de predecir y clasificar el nivel de contaminación en diferentes áreas geográficas es crucial para implementar medidas efectivas de mitigación y control. En este contexto, el modelado predictivo se presenta como una herramienta prometedora para abordar este desafío.\\

El objetivo de este estudio es desarrollar un modelo predictivo capaz de clasificar el nivel de contaminación en base a datos recopilados de diferentes fuentes y variables ambientales relevantes. Para lograrlo, se emplea la librería LazyPredict de Python, que permite evaluar diferentes modelos de clasificación y determinar cuál presenta la mejor exactitud en base a un conjunto de datos de muestra.\\

En este informe, se presenta el proceso de modelado predictivo realizado utilizando un modelo basado en Árbol de decisión (DecisionTreeClassifier) y se analizan las métricas obtenidas para evaluar su desempeño. Asimismo, se explora la utilización de la validación cruzada para obtener una evaluación más robusta del modelo.\\

Las limitaciones identificadas durante el estudio y las sugerencias de trabajos futuros se discuten con el fin de mejorar y ampliar el modelo predictivo en futuras investigaciones. En última instancia, se espera que este estudio contribuya al desarrollo de herramientas efectivas para la clasificación y predicción del nivel de contaminación, permitiendo tomar decisiones informadas y implementar medidas de protección ambiental más eficientes.\\
