%!TEX root = main.tex
\section{Limitaciones}

A pesar de los buenos resultados obtenidos en el modelado predictivo para clasificar el nivel de contaminación, es importante tener en cuenta algunas limitaciones:

\begin{itemize}
    \item \textbf{Tamaño del conjunto de datos:} Aunque se trabajó con un conjunto de datos de tamaño considerable, es posible que un conjunto de datos aún más grande pudiera haber proporcionado resultados más robustos y precisos.
    \item \textbf{Dependencia de los datos de entrada:} El modelo de Árbol de decisión utilizado en este estudio asume una dependencia lineal entre las características de entrada y la variable objetivo. Sin embargo, en la realidad, la relación puede ser más compleja y no lineal, lo que podría afectar la precisión del modelo.
    \item \textbf{Falta de características relevantes:} Es posible que el conjunto de datos utilizado no incluya todas las características relevantes para predecir el nivel de contaminación de manera óptima. La adición de más características podría mejorar aún más el rendimiento del modelo.
\end{itemize}
