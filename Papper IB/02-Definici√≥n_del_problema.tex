%!TEX root = main.tex
\section{Definición del problema}

La contaminación del aire es un problema importante que enfrentan ciudades alrededor del mundo, y Beijing, China, no es una excepción. La predicción precisa de la calidad del aire es esencial para mitigar los efectos adversos en la salud humana y el medio ambiente. En este contexto, el presente proyecto tiene como objetivo principal desarrollar un modelo de Machine Learning que pueda predecir el Índice de Calidad del Aire (AQI) en Beijing.

El AQI se basa en el nivel de cinco contaminantes atmosféricos, a saber, dióxido de azufre (SO2), dióxido de nitrógeno (NO2), partículas suspendidas (PM10), monóxido de carbono (CO) y ozono (O3). Cada uno de estos contaminantes se mide de manera diferente, con algunos calculados como un promedio diario y otros como un promedio por hora. El AQI final para un día específico se calcula como la puntuación más alta de estos cinco contaminantes.

Para llevar a cabo este objetivo, se utilizará el conjunto de datos "Beijing Multi-Site Air-Quality Data Data Set", que incluye datos por hora de contaminantes atmosféricos de 12 sitios de monitoreo de calidad del aire controlados a nivel nacional, desde el 1 de marzo de 2013 hasta el 28 de febrero de 2017.

Inicialmente, el proyecto comenzará con el análisis y modelado de los datos de un solo sitio de monitoreo. Sin embargo, si es posible, el estudio se ampliará para incluir todos los sitios de monitoreo. Se espera que este enfoque ayude a generar un modelo de predicción robusto y generalizable para el AQI en Beijing. El proyecto también permitirá a los estudiantes aumentar el valor de su trabajo incluyendo otras tareas relevantes en el proceso de análisis y modelado.
