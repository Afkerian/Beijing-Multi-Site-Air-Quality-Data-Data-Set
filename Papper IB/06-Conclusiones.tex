%!TEX root = main.tex
\section{Conclusiones}

En este estudio, se desarrolló un modelo predictivo utilizando la librería LazyPredict de Python y se obtuvieron resultados satisfactorios en la clasificación del nivel de contaminación. El modelo de Árbol de decisión alcanzó una precisión del 99.09\%, logrando clasificar correctamente la mayoría de las instancias en el conjunto de datos. 

El reporte de clasificación mostró valores elevados de recall y F1-Score para todas las clases, indicando una buena capacidad del modelo para identificar correctamente las instancias positivas y negativas. Además, el modelo fue evaluado en un conjunto de datos de tamaño considerable, lo que aumenta la confianza en su rendimiento.

Si bien existen limitaciones en cuanto al tamaño del conjunto de datos y la dependencia lineal asumida por el modelo de Árbol de decisión, los resultados obtenidos son alentadores y sugieren que el modelo puede ser útil en la clasificación del nivel de contaminación.

En cuanto a los trabajos futuros, se recomienda la recopilación de datos adicionales, la exploración de otros algoritmos de clasificación y el uso de enfoques de aprendizaje automático más avanzados para mejorar la precisión del modelo.

En resumen, este estudio demuestra la viabilidad de utilizar técnicas de modelado predictivo para la clasificación del nivel de contaminación. Los resultados obtenidos y las sugerencias de trabajos futuros pueden ser útiles para mejorar y ampliar el alcance de este enfoque en futuras investigaciones.

